\documentclass[12pt, letterpaper]{amsart}
\usepackage[left=1in,right=1in,bottom=1in,top=1in]{geometry}
\usepackage{amsfonts}
\usepackage{amsmath, amssymb}
\usepackage[font=small,labelfont=bf]{caption}
\usepackage[pdfpagelabels,hyperindex,colorlinks=true,linkcolor=blue,urlcolor=magenta,citecolor=green]{hyperref}
\usepackage{amsthm}
\usepackage{float}
\usepackage{mathrsfs}
\usepackage{colonequals}
\usepackage{natbib}

\hypersetup{
pdftitle={Unit test package documentation},
pdfsubject={Mathematics},
pdfauthor={Petro Kolosov},
pdfkeywords={Wolfram mathematica, Wolfram Language}
}


\newtheorem{thm}{Theorem}[section]
\newtheorem{cor}[thm]{Corollary}
\newtheorem{prop}[thm]{Proposition}
\newtheorem{lem}[thm]{Lemma}
\newtheorem{conj}[thm]{Conjecture}
\newtheorem{quest}[thm]{Question}
\newtheorem{ppty}[thm]{Property}
\newtheorem{ppties}[thm]{Properties}
\newtheorem{claim}[thm]{Claim}

\theoremstyle{definition}
\newtheorem{defn}[thm]{Definition}
\newtheorem{defns}[thm]{Definitions}
\newtheorem{con}[thm]{Construction}
\newtheorem{exmp}[thm]{Example}
\newtheorem{exmps}[thm]{Examples}
\newtheorem{notn}[thm]{Notation}
\newtheorem{notns}[thm]{Notations}
\newtheorem{addm}[thm]{Addendum}
\newtheorem{exer}[thm]{Exercise}
\newtheorem{limit}[thm]{Limitation}

\theoremstyle{remark}
\newtheorem{rem}[thm]{Remark}
\newtheorem{rems}[thm]{Remarks}
\newtheorem{warn}[thm]{Warning}
\newtheorem{sch}[thm]{Scholium}

\newenvironment{myitemize}
{ \begin{itemize}
    \setlength{\itemsep}{4pt}
    \setlength{\parskip}{4pt}
    \setlength{\parsep}{4pt}     }
{ \end{itemize}                  }

\makeatletter
\let\c@equation\c@thm
\raggedbottom
\makeatother
\numberwithin{equation}{section}
%--------Meta Data: Fill in your info------
%&#1043;&#1086;&#1089;&#1087;&#1086;&#1076;&#1080;, &#1048;&#1080;&#1089;&#1091;&#1089;&#1077; &#1061;&#1088;&#1080;&#1089;&#1090;&#1077;, &#1057;&#1099;&#1085;&#1077; &#1041;&#1086;&#1078;&#1080;&#1081;, &#1084;&#1086;&#1083;&#1080;&#1090;&#1074;&#1072;&#1084;&#1080; &#1055;&#1088;&#1077;&#1095;&#1080;&#1089;&#1090;&#1099;&#1103; &#1058;&#1074;&#1086;&#1077;&#1103; &#1052;&#1072;&#1090;&#1077;&#1088;&#1080; &#1080; &#1042;&#1089;&#1077;&#1093; &#1057;&#1074;&#1103;&#1090;&#1099;&#1093; &#1058;&#1074;&#1086;&#1080;&#1093;, &#1087;&#1086;&#1084;&#1080;&#1083;&#1091;&#1081; &#1085;&#1072;&#1089;. &#1040;&#1084;&#1080;&#1085;&#1100;.
\title{Unit test package documentation}
\usepackage{microtype}
\begin{document}

\begin{abstract}
This \verb".pdf" file represents a documentation for Unit Tests package in Wolfram Mathematica, which related to the math. research 
\href{https://arxiv.org/abs/1603.02468}{\textsf{https://arxiv.org/abs/1603.02468}}.
\end{abstract}
\maketitle
\tableofcontents
\section*{Installation and execution}
Prior the unit test execution, one has to download the packages \verb"main_definitions.m" and \verb"unit_tests_package.m" from \href{https://github.com/KolosovPetro/research_unit_tests}{\textsf{https://github.com/kolosovpetro/research\_unit\_tests}}. Package \verb"unit_tests_package.m" is dependent on \verb"main_definitions.m". Open these packages in Mathematica and instal by clicking \verb"File -> Install...", click \verb"Source" and choose corresponding file in dropped menu. Perform installation for two packages separately. Then recall the packages \verb"main_definitions.m" and \verb"unit_tests_package.m" in Mathematica notebook using the commands
\begin{center}
\textsf{Needs["MainDefinitions`"]}
\end{center}
\begin{center}
\textsf{Needs["UnitTests`"]}
\end{center}
Read also \href{http://support.wolfram.com/kb/5648}{\textsf{http://support.wolfram.com/kb/5648}}.

Each unit test is inside the package \verb"unit_tests_package.m", each unit test to be recalled by a single command without any inputs, for example, unit test 1 to be recalled by the command \verb"UnitTest1" to the Mathematica console. Any unit test conditions (iteration limits, etc) to be changed directly inside the package  \verb"unit_tests_package.m".
\section*{Unit test 1}
Command \verb"UnitTest1" to Mathematica console verifies an identity 
\begin{equation*}
\mathbf{P}^{m}_{a,b}(n) = \mathbf{P}^{m}_{b}(n)-\mathbf{P}^{m}_{a}(n)
\end{equation*}
\section*{Unit test 2}
Command \verb"UnitTest2" to Mathematica console verifies an identity 
\begin{equation*}
\mathbf{P}^{m}_{a,b}(n) = \sum_{k}\mathbf{X}^{m}_{k}(a,b) (-1)^{m-k} n^k
\end{equation*}
\section*{Unit test 3}
Command \verb"UnitTest3" to Mathematica console verifies an identity
\begin{equation*}
\mathbf{X}^{m}_{k}(a,b)=\mathbf{X}^{m}_{k}(b) - \mathbf{X}^{m}_{k}(a)
\end{equation*}
\section*{Unit test 4}
Command \verb"UnitTest4" to Mathematica console verifies an identity
\begin{equation*}
\mathbf{X}^{m}_{t}(a,b) = (-1)^m \sum_{k=1}^{2 m - t + 1}\mathbf{H}_{m,t}(k) (b^k - a^k)
\end{equation*}
\section*{Unit test 5}
Command \verb"UnitTest5" to Mathematica console verifies an identity
\begin{equation*}
\mathbf{P}^{m}_{a,b}(n) = \sum_{k} (-1)^{2m-k} \sum_{\ell=1}^{2m-k+1} \mathbf{H}_{m,k}(\ell)(b^\ell - a^\ell)n^k
\end{equation*}
\section*{Unit test 6}
Command \verb"UnitTest6" to Mathematica console verifies an identity
\begin{equation*}
\mathbf{Q}^{r}_{a,b}(n) = \mathbf{Q}^{r}_{b}(n) - \mathbf{Q}^{r}_{a}(n)
\end{equation*}
\section*{Unit test 7}
Command \verb"UnitTest7" to Mathematica console verifies an identity
\begin{equation*}
(f_{t}^{r} \ast f_{t}^{r})[n]  = \mathbf{Q}_{t,n-t+1}^r(n), \quad n\geq 1.
\end{equation*}
\section*{Unit test 8}
Command \verb"UnitTest8" to Mathematica console verifies an identity
\begin{equation*}
n^{2m+1} + 1 = \sum_{r\geq 0} \mathbf{A}_{m,r} (f_{0}^{r} \ast f_{0}^{r})[n], \quad n>0, \quad n\in\mathbb{N}
\end{equation*}
\section*{Unit test 9}
Command \verb"UnitTest9" to Mathematica console verifies an identity
\begin{equation*}
n^{2m+1} - 1 = \sum_{r\geq 0} \mathbf{A}_{m,r} (f_{1}^{r} \ast f_{1}^{r})[n], \quad n>0, \quad n\in\mathbb{N}
\end{equation*}
\section*{Unit test 10}
Command \verb"UnitTest10" to Mathematica console verifies an identity
\begin{equation*}
\mathbf{X}^{m}_{t}(a,b) = (-1)^m \sum_{j=t}^m \mathbf{A}_{m,j} (-1)^j \binom{j}{t} (S_{2j-t} (b) - S_{2j-t} (a))
\end{equation*}
\section*{Unit test 11}
Command \verb"UnitTest11" to Mathematica console verifies an identity
\begin{equation*}
\mathbf{P}^{m}_{a,b}(n) = \sum_{k} \sum\limits_{j\geq k}(-1)^{2m+j-k} \mathbf{A}_{m,j} \binom{j}{k}(S_{2j-k}(b)-S_{2j-k}(a)) n^k
\end{equation*}
\section*{Unit test 12}
Command \verb"UnitTest12" to Mathematica console verifies an identity
\begin{equation*}
\mathbf{P}^{m}_{a+b}(a+b) \equiv \sum_{k} \binom{2m+1}{k} a^{2m+1-k} b^k
\end{equation*}
\section*{Unit test 13}
Command \verb"UnitTest13" to Mathematica console verifies an identity
\begin{equation*}
n^{2m+1} = \mathbf{P}^{m}_{n}(n)
\end{equation*}
\section*{Unit test 14}
Command \verb"UnitTest14" to Mathematica console verifies an identity
\begin{equation*}
n^s
= n^{[s \ \mathrm{is} \ \mathrm{even}]} \mathbf{P}^{\lfloor \tfrac{s-1}{2} \rfloor}_{n}(n)
\end{equation*}
\section*{Unit test 15}
Command \verb"UnitTest15" to Mathematica console verifies an identity
\begin{equation*}
\mathbf{P}^{m}_{t,n-t+1}(n)
=\sum\limits_{r}\mathbf{A}_{m,r} \mathbf{Q}_{t,n-t+1}^r(n)
\equiv \sum\limits_{r}\mathbf{A}_{m,r} (f_{t}^{r} \ast f_{t}^{r})[n], \quad n\geq 1.
\end{equation*}
\section*{Unit test 16}
Command \verb"UnitTest16" to Mathematica console verifies an identity
\begin{equation*}
\mathbf{P}^{m}_{n}(n) - \mathbf{P}^{m}_{1,n+1}(n) = 1
\end{equation*}
\end{document}
