\documentclass[12pt, letterpaper]{amsart}
\usepackage[left=1in,right=1in,bottom=1in,top=1in]{geometry}
\usepackage{amsfonts}
\usepackage{amsmath, amssymb}
\usepackage[font=small,labelfont=bf]{caption}
\usepackage[pdfpagelabels,hyperindex,colorlinks=true,linkcolor=blue,urlcolor=magenta,citecolor=green]{hyperref}
\usepackage{amsthm}
\usepackage{float}
\usepackage{mathrsfs}
\usepackage{colonequals}

\newenvironment{myitemize}
{ \begin{itemize}
    \setlength{\itemsep}{4pt}
    \setlength{\parskip}{4pt}
    \setlength{\parsep}{4pt}     }
{ \end{itemize}                  }

\newcommand \power [2]{\langle #1 \rangle^{#2}}
\newcommand \StrilingSecond [2]{{#1} \brace {#2}}
\newcommand \multifoldsum [2][x]{#1_1 + #1_2 + \cdots + #1_{#2}}
\newcommand \curvepower [2]{\{ #1 \}^{#2}}
\newcommand \iverson[1][s]{[#1 \ \mathrm{is} \ \mathrm{even}]}
\newcommand \floor [1]{\lfloor #1 \rfloor}
\newcommand \coeffA [3][A]{\texttt{#1[#2,#3]}}
\newcommand \coeffH [4][H]{\texttt{#1[#2,#3,#4]}}
\newcommand \coeffX [4][X]{\texttt{#1[#2,#3,#4]}}
\newcommand \commonPowSum [3][S]{\mathbf{#1}_{#2}(#3)}
\newcommand \mynotationFixed [4][P]{\texttt{#1[#2,#3,#4]}}
\newcommand \polynomLFixed [4][L]{\texttt{#1[#2,#3,#4]}}
\newcommand \convSumFixed [4][C]{\mathbf{#1}^{#2}_{#3}(#4)}

\newtheorem{thm}{Theorem}[section]
\newtheorem{cor}[thm]{Corollary}
\newtheorem{prop}[thm]{Proposition}
\newtheorem{lem}[thm]{Lemma}
\newtheorem{conj}[thm]{Conjecture}
\newtheorem{quest}[thm]{Question}
\newtheorem{ppty}[thm]{Property}
\newtheorem{ppties}[thm]{Properties}
\newtheorem{claim}[thm]{Claim}

\theoremstyle{definition}
\newtheorem{defn}[thm]{Definition}
\newtheorem{defns}[thm]{Definitions}
\newtheorem{con}[thm]{Construction}
\newtheorem{exmp}[thm]{Example}
\newtheorem{exmps}[thm]{Examples}
\newtheorem{notn}[thm]{Notation}
\newtheorem{notns}[thm]{Notations}
\newtheorem{addm}[thm]{Addendum}
\newtheorem{exer}[thm]{Exercise}
\newtheorem{limit}[thm]{Limitation}

\theoremstyle{remark}
\newtheorem{rem}[thm]{Remark}
\newtheorem{rems}[thm]{Remarks}
\newtheorem{warn}[thm]{Warning}
\newtheorem{sch}[thm]{Scholium}

\makeatletter
\let\c@equation\c@thm
\raggedbottom
\makeatother
\numberwithin{equation}{section}
%--------Meta Data: Fill in your info------
\title[main\_definitions.m package documentation]
{main\_definitions.m package documentation}
\hypersetup{
pdftitle={On the link between Binomial Theorem and Discrete Convolution of  Power Function},
pdfsubject={Discrete Mathematics, Number Theory, Combinatorics},
pdfkeywords={Binomial theorem, Discrete convolution, Power function, Polynomials,  Convolution, Multinomial theorem, Binomial coefficient, Bernoulli number, Pascal's triangle, Faulhaber's formula, Power sum, Worpitzky identity, Macaulay's method, Macaulay brackets, Iverson's bracket, Binomial expansion}
}
\let\CheckCommand\providecommand
\usepackage{microtype}
\begin{document}
\maketitle
\tableofcontents
\section{Introduction}
This file represents a documentation for \texttt{main\_definitions.m} Mathematica package. To get started proceed to GitHub repository \href{https://github.com/KolosovPetro/arXiv1603.02468-Mathematica-Implementations}
{\textsf{https://github.com/KolosovPetro/arXiv1603.02468-Mathematica-Implementations}}, pull it, and find the package \texttt{main\_definitions.m}. This package doesn't have any dependencies on other Mathematica packages. To get started simply install it to your Mathematica by clicking \verb"File -> Install...", click \verb"Source" and choose corresponding file in dropped menu. Then recall the package \verb"main_definitions.m" in Mathematica notebook using the command

\begin{center}
\texttt{Needs["MainDefinitions`"]}
\end{center}
Example of a Mathematica notebook where \texttt{main\_definitions.m} recalled is available in github repository as well.
Read also \href{http://support.wolfram.com/kb/5648}{\textsf{http://support.wolfram.com/kb/5648}}.

\section{Functions in package \texttt{main\_definitions.m}}
We now set the following notation, which remains fixed for the remainder of this paper:
\begin{myitemize}
\item $\coeffA{m}{r}$ is a real coefficient defined recursively
\begin{equation*}
\coeffA{m}{r} \colonequals
\begin{cases}
(2r+1)\binom{2r}{r} & \mathrm{if } \ r=m \\
(2r+1)\binom{2r}{r} \sum_{d=2r+1}^{m} \coeffA{m}{d} \binom{d}{2r+1} \frac{(-1)^{d-1}}{d-r} B_{2d-2r} & \mathrm{if } \ 0 \leq r < m \\
0 & \mathrm{if } \ r<0 \ \mathrm{or } \ r>m
\end{cases}
\end{equation*}
where $B_t$ are Bernoulli numbers.
\item $\polynomLFixed{m}{n}{k}$ is polynomial of degree $2m$ in $n,k$
\begin{equation*}
\polynomLFixed{m}{n}{k} \colonequals \sum_{r=0}^{\texttt{m}} \coeffA{m}{r} \texttt{k}^r \texttt{(n-k)}^r
\end{equation*}
\item $\mynotationFixed{m}{n}{b}$ is polynomial of degree $2m+1$ in $b, n$
\begin{equation*}
\label{p_definition}
\mynotationFixed{m}{n}{b} \colonequals \sum_{\texttt{k}=0}^{\texttt{b}-1} \polynomLFixed{m}{n}{k}
\end{equation*}
\item $\texttt{ConvolveSum[n, r, b]}$ is a convolutional power sum
\begin{equation*}
\label{q_definition}
\texttt{ConvolveSum[n, r, b]} \colonequals \sum_{k=0}^{b-1} k^r(n-k)^r
\end{equation*}
\item $\coeffH{m}{t}{b}$ is a real coefficient defined as
\begin{equation*}
\coeffH{m}{t}{b} \colonequals \sum_{j=\texttt{t}}^{\texttt{m}} \binom{j}{\texttt{t}} \coeffA{m}{j} \frac{(-1)^j}{2j-\texttt{t}+1} \binom{2j-\texttt{t}+1}{\texttt{b}} B_{2j-\texttt{t}+1-\texttt{b}}
\end{equation*}
\item $\coeffX{m}{t}{j}$ is polynomial of degree $2m-t$ in $b$
\begin{equation*}
\coeffX{m}{t}{j} \colonequals (-1)^m \sum_{k=1}^{2m-t+1} \coeffH{m}{t}{k} \cdot \texttt{j}^k
\end{equation*}
\item $\texttt{S[p, n]}$ is a common power sum
\begin{equation*}
\texttt{S[p, n]} \colonequals \sum_{k=0}^{n-1} k^p
\end{equation*}
\item $\texttt{MacaulayPow[x,n,a]}$ is powered Macaulay bracket
\begin{equation*}
\label{MacaulayPow}
\texttt{MacaulayPow[x, n, a]} \colonequals
\begin{cases}
(x-a)^n, &\quad x\geq a \\
0, &\quad \mathrm{otherwise}
\end{cases} \quad a\in\mathbb{Z}
\end{equation*}
\item $\texttt{MacaulayPowStrict[x,n,a]}$ is powered Macaulay bracket
\begin{equation*}
\label{MacaulayPowStrict}
\texttt{MacaulayPowStrict[x,n,a]} \colonequals
\begin{cases}
(x-a)^n, &\quad x>a \\
0, &\quad \mathrm{otherwise}
\end{cases}  \quad a\in\mathbb{Z}
\end{equation*}
\end{myitemize}
\section{Examples of tests}
For example, we can verify the identity $n^{2m+1}=\mynotationFixed{m}{n}{n}$ as follows
$$\mynotationFixed{m}{n}{n}=n^{2m+1}$$

All results of the manuscript \texttt{https://arxiv.org/abs/1603.02468} can be verified similarly.
\end{document}
